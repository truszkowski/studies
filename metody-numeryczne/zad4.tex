\documentclass[12pt,a4paper,twoside]{article}
\RequirePackage[T1]{fontenc}
\RequirePackage{times}
\RequirePackage[latin2]{inputenc} 
\RequirePackage[polish]{babel}
\RequirePackage{comment} 
\RequirePackage{a4wide} 
\RequirePackage{longtable}
\RequirePackage{multicol} 
\RequirePackage{fancyhdr}
\RequirePackage{exscale}
\RequirePackage{verbatim}
\RequirePackage{amsmath}
\RequirePackage{url}
\RequirePackage[dvips,dvipdf]{graphicx}
\author{Piotr Truszkowski}
\title{Metody numeryczne --- zadanie 4}

\begin{document}

\pagestyle{fancy}
\fancyhf{}
\renewcommand{\headrulewidth}{0.5pt}
\renewcommand{\footrulewidth}{0pt}
\addtolength{\headheight}{0.5pt}
\lhead[\emph{Piotr Truszkowski}]{\emph{Piotr Truszkowski}}
\rhead[\thepage]{\thepage}

\section*{Zadanie 1}

Zauwazmy, ze funkcja $\phi_{i}(x)$ jest nie r�zniczkowalna, lecz na 
potrzeby zadania mozemy ja na sile zr�zniczkowac.
$\phi_{i}'(x)$ wyniesie $1$ gdy $x \in [(i-1)h,ih)$ 
wyniesie $-1$ gdy $x \in [ih,(i+1)h]$ oraz $0$ wpp.
Liczenie kwadratury mozemy, wiec znaczaco uproscic.
Po pierwsze wystarczy uwzgledniac pjedynie przedialy gdzie $\phi_{i}(x)$ osiaga 
nie zerowa wartosc. Po drugie mozna zauwazyc, ze poprzez wyliczenie  kwadratury
dla $i$ na odcinku $[ih,(i+1)h]$ mozna ja uzyc do wyliczenia kwadratury 
dla $i+1$ na odcinku tym samym, jedynie z innym znakiem. Oszczednosc obliczeniowa 
dwukrotna. 
\\
W przypadku korzystania z biblioteki GSL w przypadku drugiej funkcji mo�na 
uzyc specjalnej kwadratury dla funkcji silnie oscylujacych  (oddzielnie dlasin i cos).
\\
\\
Uklad r�wnan $G(U)h* - F$ jest ukladem r�wnan z macierza tr�jdiagonalna symetryczna.
GSL dysponuje specjalna procedura do wyliczania ukladu z takimi wlasnie macierzami.
Wystarczy tylko podac za argumenty wektor diagonali, poddiagonali.
\\
\\
Dla dosc duzych danych $N = K = 10^6, \epsilon = 10^{-6}$ GSL protestuje, 
oznajmiajac, iz nie jest w stanie  policzyc kwadratury z zadana dokladnoscia.
Dla $\epsilon = 10^{-4}$ jest juz w stanie.

\end{document}
